\documentclass[12pt]{article}

% Change "review" to "final" to generate the final (sometimes called camera-ready) version.
% Change to "preprint" to generate a non-anonymous version with page numbers.
\usepackage[preprint]{acl}

% Standard package includes
\usepackage{times}
\usepackage{latexsym}

% For proper rendering and hyphenation of words containing Latin characters (including in bib files)
\usepackage[T1]{fontenc}
% For Vietnamese characters
% \usepackage[T5]{fontenc}
% See https://www.latex-project.org/help/documentation/encguide.pdf for other character sets

% This assumes your files are encoded as UTF8
\usepackage[utf8]{inputenc}

% This is not strictly necessary, and may be commented out,
% but it will improve the layout of the manuscript,
% and will typically save some space.
\usepackage{microtype}

% This is also not strictly necessary, and may be commented out.
% However, it will improve the aesthetics of text in
% the typewriter font.
\usepackage{inconsolata}

%Including images in your LaTeX document requires adding
%additional package(s)
\usepackage{graphicx}
% Allow tables to span multiple pages when necessary
\usepackage{longtable}
\usepackage{booktabs}

% If the title and author information does not fit in the area allocated, uncomment the following
%
%\setlength\titlebox{<dim>}
%
% and set <dim> to something 5cm or larger.


\title{Instructions for *ACL Proceedings}

% Author information can be set in various styles:
% For several authors from the same institution:
% \author{Author 1 \and ... \and Author n \\
%         Address line \\ ... \\ Address line}
% if the names do not fit well on one line use
%         Author 1 \\ {\bf Author 2} \\ ... \\ {\bf Author n} \\
% For authors from different institutions:
% \author{Author 1 \\ Address line \\  ... \\ Address line
%         \And  ... \And
%         Author n \\ Address line \\ ... \\ Address line}
% To start a separate ``row'' of authors use \AND, as in
% \author{Author 1 \\ Address line \\  ... \\ Address line
%         \AND
%         Author 2 \\ Address line \\ ... \\ Address line \And
%         Author 3 \\ Address line \\ ... \\ Address line}

\author{Nour Eljadiri \\
  University of Passau \\
  INSA Lyon \\
  \texttt{mohamed.eljadiri@insa-lyon.fr} \\
}

%\author{
%  \textbf{First Author\textsuperscript{1}},
%  \textbf{Second Author\textsuperscript{1,2}},
%  \textbf{Third T. Author\textsuperscript{1}},
%  \textbf{Fourth Author\textsuperscript{1}},
%\\
%  \textbf{Fifth Author\textsuperscript{1,2}},
%  \textbf{Sixth Author\textsuperscript{1}},
%  \textbf{Seventh Author\textsuperscript{1}},
%  \textbf{Eighth Author \textsuperscript{1,2,3,4}},
%\\
%  \textbf{Ninth Author\textsuperscript{1}},
%  \textbf{Tenth Author\textsuperscript{1}},
%  \textbf{Eleventh E. Author\textsuperscript{1,2,3,4,5}},
%  \textbf{Twelfth Author\textsuperscript{1}},
%\\
%  \textbf{Thirteenth Author\textsuperscript{3}},
%  \textbf{Fourteenth F. Author\textsuperscript{2,4}},
%  \textbf{Fifteenth Author\textsuperscript{1}},
%  \textbf{Sixteenth Author\textsuperscript{1}},
%\\
%  \textbf{Seventeenth S. Author\textsuperscript{4,5}},
%  \textbf{Eighteenth Author\textsuperscript{3,4}},
%  \textbf{Nineteenth N. Author\textsuperscript{2,5}},
%  \textbf{Twentieth Author\textsuperscript{1}}
%\\
%\\
%  \textsuperscript{1}Affiliation 1,
%  \textsuperscript{2}Affiliation 2,
%  \textsuperscript{3}Affiliation 3,
%  \textsuperscript{4}Affiliation 4,
%  \textsuperscript{5}Affiliation 5
%\\
%  \small{
%    \textbf{Correspondence:} \href{mailto:email@domain}{email@domain}
%  }
%}

\begin{document}

\maketitle

\begin{abstract}
    The task of assigning multiple, hierarchically-structured labels to text documents, known as Hierarchical Multi-Label Classification (HMLC), is critical in domains from scientific archiving to legal analysis. This review traces the methodological evolution of HMLC, beginning with foundational problem transformation methods like Binary Relevance and Classifier Chains, which primarily address the challenge of label correlation. We then examine the paradigm shift introduced by pre-trained Transformers, dissecting the dichotomy between local, top-down approaches prone to error propagation and global, hierarchy-aware models that integrate structural constraints via specialized loss functions or Graph Neural Networks. Finally, we explore the current frontier, where Large Language Models (LLMs) are reframing the task through generative paradigms, enabled by techniques such as LLM-powered data augmentation, instruction fine-tuning, and Parameter-Efficient Fine-Tuning (PEFT). This narrative highlights a progression towards increasingly sophisticated methods for embedding hierarchical prior knowledge into statistical models.
\end{abstract}

\section{Introduction}

Text Classification (TC) stands as one of the most foundational and widely researched tasks within the domain of Natural Language Processing (NLP). \cite{Zangari2024}  In its most common formulation, TC involves supervised learning algorithms designed to map a given piece of text, or document, to a predefined set of labels or categories. \cite{Zangari2024} Historically, this has often been a multiclass problem, where each document is assigned to exactly one class from a set of mutually exclusive options. However, the increasing complexity and richness of information in modern digital text have rendered this single-label paradigm insufficient.\cite{Hu2025}  Many documents, from news articles and scientific papers to legal filings and product descriptions, simultaneously encompass multiple topics or themes. \cite{TidakeSane2018}

This reality gave rise to Multi-Label classification (MLC), a more challenging variant of text classification where each data sample can be associated with one or multiple labels simultaneously.\cite{TidakeSane2018} The core challenge in MLC, which distinguishes it from simply running multiple independent binary classifiers, is the presence of correlations between labels \cite{TidakeSane2018}, that is, the assignment of one label often provides strong statistical evidence for or against the assignment of another, and effectively modeling these inter-label dependencies has become a central focus of research in the field. \cite{Huang2024, TidakeSane2018}

Formally, in MLC, the goal is to learn a function $f: X \rightarrow 2^L$ that maps an instance $x \in X$ to a subset of labels $Y \subseteq L$, where $L = \{l_1, l_2, \ldots, l_L\}$ is the finite set of all possible labels. The number of labels associated with an instance is not fixed and can vary. \cite{TidakeSane2018}

Hierarchical Multi-Label Text Classification (HMLC), the primary subject of this review, introduces a further layer of complexity and structure to the MLC problem. HMLC is defined as a classification task where instances may not only belong to multiple classes simultaneously, but where these classes are themselves organized within a predefined hierarchy \cite{liu2023recentadvanceshierarchicalmultilabel}. This hierarchical structure, typically represented as a tree or a Directed Acyclic Graph (DAG), formalizes the relationships among the labels, arranging them from broader, coarse-grained categories at higher levels to more specific, fine-grained ones. \cite{liu2023recentadvanceshierarchicalmultilabel}

This structured approach is particularly relevant for analyzing the sophisticated communication strategies found in online media. The rapid spread of online news has increased exposure to deceptive narratives and manipulation attempts, especially during major crisis events like geopolitical conflicts. To support research in this area, tasks such as SemEval-2025 Task 10 have been established, focusing on automated narrative classification \cite{semeval2025task10}. The goal is to categorize news articles by assigning them multiple labels from a two-level taxonomy of predefined narratives and subnarratives. Addressing this HMLC problem requires models that can understand nuanced content while respecting the explicit hierarchical dependencies between labels.

The task of narrative detection in text is an ideal application for HMLC. Narratives are structured frameworks of meaning that shape the interpretation of events and issues. A single text can invoke multiple, often nested, narratives. For example, a news report on an international incident might simultaneously employ a broad "National Security" narrative, a more specific "Foreign Aggression" sub-narrative, and a granular "Economic Sanctions" micro-narrative. An HMLC framework can model this structure, capturing both the multiple narrative elements present and their hierarchical relationships.

This review will trace the methodological evolution of HMLC. We begin by examining foundational problem transformation techniques such as Binary Relevance \cite{zhang_binary_2018}, Classifier Chains \cite{li_relative_2024, weng_label_2020}, and the Label Powerset method \cite{shan_co-learning_2018}. We then transition to the current state-of-the-art, dominated by deep learning models leveraging Transformer architectures like BERT \cite{devlin_bert_2019} and its multilingual variants such as XLM-RoBERTa \cite{conneau_unsupervised_2020}. Finally, we will cover specialized strategies like hierarchical classification models \cite{sadat_hierarchical_2022} and graph-based methods \cite{peng_hierarchical_2021, gong_hierarchical_2020} designed to explicitly model the structured taxonomies inherent to HMLC.

\chapter{Foundational paradigms of MLC}
The main challenge in multi-label classification (MLC) was how to adapt algorithms designed for single-label (binary or multiclass) problems to handle multiple labels per instance. The core issue these methods faced was the presence of label correlations: in real-world data, labels are often not independent. For example, a news article tagged "Politics" is much more likely to also be tagged "Elections" than "Sports." Treating each label as a separate binary problem ignores these dependencies, potentially leading to suboptimal predictions.

Classical MLC methods can be seen as different strategies for balancing computational simplicity with the need to model label correlations. Some methods, like Binary Relevance, treat each label independently for simplicity, but this can miss important relationships between labels. Others, such as Classifier Chains or Label Powerset, explicitly model these correlations, but at the cost of increased computational complexity. The choice of method reflects a trade-off between efficiency and the ability to capture the true structure of the data.

To address this, early research focused on a family of techniques known as problem transformation methods, which decompose the multi-label task into one or more single-label problems. These methods are algorithm-independent, allowing any standard classifier to be applied. The three canonical approaches represent distinct strategies for managing label dependencies.

\section{Binary Relevance}
Binary Relevance (BR) is the most intuitive approach. It decomposes the MLC problem with a label set of size $|\mathcal{L}|$ into independent binary classification problems. For each label, a separate classifier is trained to predict its presence or absence, effectively ignoring all other labels. \cite{zhang_binary_2018} The primary advantage of BR is its simplicity and efficiency, as the classifiers can be trained in parallel. Its main drawback, however, is the foundational label independence assumption, which completely disregards label correlations and can lead to lower predictive accuracy and logically incoherent label set predictions. \cite{Sucar2014Multi-label}

\section{Label Powerset}
In contrast to decompositional methods, the Label Powerset (LP) method reframes the entire problem at once. It converts the multi-label task into a standard multi-class problem by mapping each distinct set of co-occurring labels to a single, unique class. For instance, the label sets `{Politics, Elections}` and `{Sports, Weather}` would become two separate classes for a multi-class classifier to learn. \cite{Read2011}

The main advantage of this approach is its ability to perfectly model the dependencies between labels for all combinations it has seen, as these correlations are baked into the class definitions \cite{Sucar2014Multi-label}. However, this strategy is often impractical. The number of potential classes can become unmanageably large as the label set grows, a problem known as combinatorial explosion. This leads to a highly sparse class distribution where many label sets appear only a few times, making it difficult to train a robust model \cite{Cherman2011}. Critically, the LP method cannot generalize to predict any combination of labels that did not appear in the training data.

\section{Classifier Chains}
The Classifier Chains (CC) method was proposed as a novel approach to overcome the stark trade-off between the label-independent BR and the computationally explosive LP. It seeks to model label dependencies while maintaining the efficiency of a binary relevance framework \cite{Read2011}.

Like BR, CC trains $|\mathcal{L}|$ binary classifiers. However, instead of being independent, these classifiers are linked in a chain. The first classifier in the chain, $\mathcal{C}_1$, predicts the presence or absence of the first label. The predictions of this classifier are then used as additional features for the second classifier, $\mathcal{C}_2$, which predicts the second label. This process continues down the chain, with each classifier potentially benefiting from the predictions of all previous classifiers. This allows CC to capture label dependencies while still being relatively efficient to train. \cite{Read2011} The order of the chain can significantly impact performance, as earlier classifiers influence later ones \cite{Read2021}. To mitigate this, ensemble methods that average predictions over multiple random chain orders are often employed \cite{Sucar2014Multi-label,Zhang2018}.



\chapter{Hierarchy aware architectures}
The limitations inherent in classical problem transformation methods, particularly their struggles with large label spaces and their inability to deeply integrate structural information, precipitated a paradigm shift toward deep learning.
We will focus on how neural architectures evolved from treating the label hierarchy as a post-hoc constraint to using it as a central component of the learning process.
This evolution is characterized by a fundamental architectural divergence between local and global approaches, a conflict that was ultimately resolved through the powerful synthesis of Transformer-based text encoders and Graph Neural Networks (GNNs) for structure encoding.

\section{From flat to hierarchical models}
Early deep-learning methods for hierarchical multi-label classification often treated the task as a standard (flat) multi-label problem and ignored the label taxonomy. While these models learned strong text representations, they did not use the hierarchy's relationships. That matters because mistakes at higher levels of the taxonomy are semantically worse than small, nearby errors: for example, assigning a paper on "Quantum Mechanics" to "Arts" is much more serious than confusing "Physics" with "Chemistry." Flat models cannot distinguish these degrees of error.~\cite{xu-etal-2021-hierarchical}

The paradigm shift occurred with the recognition that the hierarchy could be treated as a feature to guide learning, rather than just an output format. By making models "hierarchy-aware," it becomes possible to share statistical strength between parent and child nodes. For example, the few training instances available for a rare, specific label like "Superstring Theory" can be supplemented by the more abundant data from its parent labels, "String Theory" and "Theoretical Physics." This is especially crucial for improving performance on the long tail of infrequent labels that characterizes most real-world HMLC datasets.~\cite{Zangari2024}

\section{Local vs global approaches}

The first generation of truly hierarchical models diverged into two main architectural philosophies: local and global. This division reflects a fundamental trade-off between capturing fine-grained, localized class relationships and maintaining a holistic, computationally tractable view of the entire label space.

\subsection{Local approaches (Top down)}
The local approach decomposes the hierarchical classification problem into a set of smaller, more manageable classification tasks distributed across the taxonomy. This is typically implemented as a top-down or "divide and conquer" strategy.

During inference, an instance is typically classified in a top-down manner. It is first evaluated by the classifier at the root; if a positive prediction is made for a node, the instance is then passed down to the classifiers of its children, and this process continues until a leaf node is reached or no further positive predictions are made.~\cite{Romero2022}. While this approach excels at capturing the specific features that distinguish between closely related sibling classes, it suffers from a critical problem: \textbf{error propagation.} A single misclassification at a higher level of the hierarchy can irreversibly steer the prediction down an incorrect path, making it impossible to classify the instance into its correct, more specific sub-categories.~\cite{Wehrmann2018}

\subsection{Global approaches (single classifier)}
In contrast, the global approach uses a single, unified model to predict all labels in the hierarchy simultaneously. This is typically framed as a large multi-label classification problem where the output layer corresponds to the entire set of labels in the taxonomy.~\cite{Wehrmann2018}

The primary advantage of the global approach is that it inherently avoids the error propagation problem of local models, as all decisions are made in parallel by a single classifier. This makes the model more robust to errors at higher levels. Furthermore, global models are often more computationally efficient, as they require training only one model instead of a potentially large cascade of local classifiers. However, early global models faced a significant challenge: they struggled to effectively incorporate the complex structural information of the entire hierarchy into a single model. By treating the problem as a flat multi-label task, they often failed to capture the nuanced, local distinctions between sibling classes and could underfit the hierarchical relationships, thereby losing the very information that hierarchical classification aims to exploit.~\cite{Wehrmann2018, zhou-etal-2020-hierarchy}

\section{Encoding the Hierarchy with Transformers and Graph Neural Networks}

The solution to the local-versus-global problem came from combining two methods: Transformer models to understand text, and Graph Neural Networks (GNNs) to understand structure. This mix made it possible to build global models that are both efficient and aware of hierarchies. \cite{wang2024graphneuralnetworkstext, zhou-etal-2020-hierarchy}

GNNs are especially useful for working with label hierarchies. \cite{li2021heterogeneous} We can represent the taxonomy as a graph, where labels are nodes and parent–child links are edges. A GNN learns label representations by passing messages between connected nodes. Each node gathers information from its parents, children, and siblings. This way, the final label embeddings capture not just their own meaning, but also their place and relationships within the whole taxonomy.

A seminal work in this domain is the Hierarchy-Aware Global Model (HiAGM) by Zhou et al. (2020). HiAGM provides a blueprint for this new class of models. Its architecture consists of two main components:~\cite{zhou-etal-2020-hierarchy}

\begin{itemize}
	\item A Text Encoder (e.g., BERT, RoBERTa) that generates a powerful contextualized representation of the input document.
	\item A Hierarchy-Aware Structure Encoder (e.g., a Bidirectional Tree-LSTM or a specialized Hierarchy-GCN) that operates on the label graph to produce hierarchy-aware label embeddings. This encoder models dependencies in both a top-down and bottom-up fashion, allowing information to flow in both directions across the hierarchy.
\end{itemize}

HiAGM further proposes two distinct fusion variants to combine text and structure features depending on the inference regime and efficiency/inductivity trade-offs.

\paragraph{HiAGM-LA (Multi-Label Attention)}
An inductive approach that uses a multi-label attention mechanism to compute document-specific label representations by attending from the text encoder outputs to the hierarchy-aware label embeddings. Because the attention weights are computed per document, the model can generalize to unseen instances without explicitly re-running a graph propagation step for each input — label embeddings can be pre-computed and stored, and the attention operation is applied at inference time.

\paragraph{HiAGM-TP (Text Feature Propagation)}
A deductive approach that propagates text features across the label hierarchy via graph propagation: document features are attached to label nodes and a GNN is run to diffuse these features through the taxonomy. This requires executing the GNN at inference time for each instance (or batch), which can be more computationally expensive but allows richer instance-specific propagation of evidence through the hierarchy.

The principles demonstrated by HiAGM have been extended and refined in subsequent work. For instance, Xu et al. (2021) proposed a framework using a loosely coupled GCN to explicitly model not only the vertical correlations (parent-child dependencies) but also the horizontal correlations (relationships between sibling nodes at the same level). This allows the model to capture, for example, that "Computer Science" and "Electrical Engineering" are more closely related to each other than to "History," even if they share the same parent `Science \& Technology'.


\chapter{The rise of Large Language Models}

The advent of Large Language Models (LLMs) has initiated another profound shift in the NLP landscape. Their advanced reasoning and generation capabilities are redefining their role in the HMLC pipeline, moving them beyond being simple end-point classifiers to becoming integral co-pilots in various stages of the workflow.

\section{Zero shot and few-shot classification}
The most immediate impact of LLMs is their remarkable ability to perform classification tasks with little to no task-specific training data. Through in-context learning, an LLM can be prompted with a task description and a few examples (few-shot) or even no examples (zero-shot) and perform hierarchical multi-label classification. This capability represents a paradigm shift for low-resource scenarios, potentially obviating the need for extensive data collection and annotation efforts; for narrative detection, one can provide definitions of the target narratives and ask the LLM to classify a new document accordingly.~\cite{wang-etal-2023-text2topic}

For instance, Eljadiri and Nurbakova (2025) demonstrate a high-performing zero-shot agentic approach to narrative classification. Their system instantiates multiple specialized LLM agents, each responsible for a binary decision on a single narrative or subnarrative label, while a meta-agent aggregates the binary outputs into final multi-label predictions. The agents are orchestrated with AutoGen and operate without task-specific fine-tuning, which enables parallel detection across the two-level taxonomy; this design yielded competitive results on the SemEval-2025 test set and highlights the practical zero-shot LLM systems for complex hierarchical tasks.~\cite{eljadiri-nurbakova-2025-team}


\section{LLMs for data augmentation}
\label{sec:llm-augmentation}

One of the most promising applications of LLMs is in addressing data scarcity and class imbalance through synthetic data generation. An LLM can be prompted to act as a domain expert and generate new text samples for underrepresented (tail) classes. This can be done by paraphrasing existing examples to increase diversity or by generating entirely new, plausible examples from scratch. This approach offers a sophisticated and flexible alternative to traditional data augmentation techniques like synonym replacement or back-translation. \cite{cegin-etal-2025-llms,glazkova2024evaluating}

In more extreme cases; for example, a scenario with complete lack of labeled data, an LLM can be used to generate a large corpus of "weak" or "silver" labels. These labels, while imperfect, can serve as a starting point for training a smaller, more efficient model. Frameworks like JSDRV by Yang et al. (2024) take this a step further, using a reinforcement learning policy to select the highest-quality LLM-generated annotations for fine-tuning, creating a self-improving annotation and training pipeline \cite{yang-etal-2024-reinforcement}.

\section{LLMs as evaluation assistants}
For extreme multi-label classification tasks with thousands of labels, manual evaluation is prohibitively expensive and time-consuming. Li et al. (2023) aimed to compare a label-ranking BiCross-Encoder against a SciBERT classifier in a very large-label setting (11,486 labels) while lacking a ready human-annotated test set. They demonstrated a practical approach by asking ChatGPT to rate the relevance of candidate labels: for each document they fed the top-10 model-predicted labels to ChatGPT and requested a 3-point relevance score (0 = irrelevant, 1 = somewhat relevant, 2 = highly relevant) along with brief explanations. The authors treated these LLM judgments as automatic annotations to compare models, then spot-checked a sample with subject-matter experts and reported about 60\% agreement on the 3-point scale (rising to 82\% when collapsing 1 and 2 into a single ``relevant'' class). They conclude that using ChatGPT as an evaluation assistant is a cost- and time-efficient way to bootstrap evaluation when comprehensive human annotation is infeasible. \cite{li-etal-2023-enhancing-extreme}



\bibliography{references, anthology}
\end{document}
